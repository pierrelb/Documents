\documentclass[11pt]{article}
\begin{document}

\title{Automatic generation of transition state geometries.}
\author{Pierre L. Bhoorasingh, Richard H. West}
\date{\today}
\maketitle

\newpage
\paragraph{Abstract}

\newpage
\section{Introduction}
The effects of global warming and rising energy costs continue to motivate work in fuel combustion. Renewable fuels and advanced engines with higher efficiencies to use these fuels are required to help quell the rising energy demands. The costs associated with engine design and fuel development require modeling solutions to reduce the steps and focus the research.

Detailed kinetic models provide insight into the complex chemistry of combustion, and can be used to develop models to aid engine design. These models can become predictive with the improvement of the methods used to generate the kinetic parameters. Rate parameters are ideally used from experimental data, but the difficulty in obtaining this data means relatively few reactions have been accurately determined. (Speak more on the difficulties)

(Speak about estimation methods)

(Introduce your work - e.g. automatic generation has been identified as a 'Grand Challenge', etc.)

\section{Methods}
\subsection{Molecular bounds matrix}
A bounds matrix defines the minimum and maximum distances in Angstroms between any two atoms. An index is associated with each atom and the row and column of the matrix with a given index represents the related atom. Atoms on different molecules can be infinitely distant so large numbers are used as their maximum distances. Their minimum distances are defined by the sum of the van der Waals radii of the relevant atoms. The range used for bonded atoms is determined by bond length values. Ranges for atoms within a molecule but not bonded to one another are determined by the knowledge that the maximum distances occur when all atoms between are aligned in a straight line, and the minimum is limited by steric hindrance and interatomic forces.

\subsection{Conformer generation}
Initial stable molecular configuration estimates were generated using RDKit. A molecular bounds matrix can also be processed by RDKit to generate and partially optimize a molecular structure within the limits of the matrix. Low level universal force fields are used to do the optimization. This enables a bounds matrix to be set and used so that reactant and product geometries can be generated so that active atoms in the reaction are positioned close to their reactive positions. Double-ended transition state search methods require this type of positioning of reactant and product geometries before they can be used effectively.

\subsection{Double-ended searches and further refinement}
The double-ended searches are so defined as both ends (reactants and products) of the reaction are required to generate the transition state geometry. Gaussian 09 (G09) and MOPAC 2012 were used for the quantum chemical calculations. The quasi-synchronous transit search in G09 and MOPAC's string-like search method were used to test the efficacy of automation with a double-ended search with semi-empirical method (PM6 in G09, PM7 in MOPAC). These semi-empirical methods were then used on the resulting transition state estimate to further refine the estimate at the level of theory. The M06-2x density functional theory with the 6-31+G (d,p) basis set was used to test the success of this automatic method at higher levels of theory. These higher level calculations were only conducted on the refined semi-empirically generated geometries in order to minimize computing costs.

\subsection{Determining reacting atom distances}
Editing bounds matrices to setup the double-ended searches requires positioning active atoms in a reactive state. How close those atoms should be positioned were initially determined by hand. A few examples for a given reaction family helped determine an approximate distance between these active atoms. For a hydrogen abstraction reaction, the active atoms are the transferred hydrogen, the atom it leaves (donor), and the atom it forms a bond with (acceptor). For the reactant distances in the double-ended search, the donor and acceptor distance had to be between 2.5 and 2.6 angstroms, and the acceptor and hydrogen distance had to be between 2.0 and 2.1 angstroms. The donor and hydrogen atoms used typical bond length distances. For the products, the same distances were used, except the donor is now the acceptor atom and the acceptor is the donor (the reaction in the reverse direction). It is possible that these distances have been successful due to the similarity in atom groups tested, but further work will be required to determine the effects of group contributions.

\section{Results and Discussion}

\section{Conclusions}

\end{document}

\bibliography{}
