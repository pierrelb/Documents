\documentclass[12pt]{article}
\begin{document}

\title{}
\author{Pierre L. Bhoorasingh}
\date{\today}
\maketitle

\newpage
\tableofcontents

\newpage

\section{Summary}

\subsection{Aims}

\section{Introduction}

\subsection{Fuels of the Future}

\subsection{Automatic Mechanism Generation}

\subsubsection{Network Generation}
Automatic network generation systems involve methods that teach chemistry to a computer. These automatic systems all have common features. They must have a method to represent atoms, bonds, and other molecular properties. There must be transformation methods to represent reactions in order to convert reactants to products and methods to determine if a molecule generated is unique from all previously generated species. These will enable the generation of all the possible reactions.

\paragraph{Possibly talk about chemical graph theory, SMILES, InChI, etc.}

\subsubsection{Thermodynamic and Kinetic Parameters}
With the chemical reaction network defined, the thermodynamic and kinetic parameters are needed to fully describe the reaction system. Also, model reduction may also be desired due to the scale of these mechanisms, and kinetic parameters are required to conduct sensitivity analyses.

Describing thermodynamic and rate parameters by hand is taxing, since reaction mechanisms can have a very large number of species and reactions. Computational methods allow for the integration of the various methods to procure and specify these parameters for such large mechanisms. A ranking has to be implemented to ensure preferred methods are attempted before the less desirable, and this ranking is intended to maximize accuracy while minimizing computational cost. There are three general sources of data, both for kinetic and thermodynamic parameters: experimental values, estimation by analogy, and estimation via quantum mechanical calculations.

Well-founded experimental values in databases are the premier data source. Not only do these parameters contain little error, but the high efficiency in extracting a value from a database suits automated systems. However, these coveted values are scare due to the system restrictions and experimental difficulties. For kinetics, most of the species in a mechanism are intermediates, making their detection and measurement and arduous task. Added to the difficulty is the complexity of the reaction environment. These two points lead to the scarcity in kinetic parameters. More data is available for thermodynamic properties, but these data are restricted to the system studied. Therefore, values for stable molecules are accessible, but species such as radical have limited data.  As such, it is not possible to derive a detailed mechanism using experimentally derived values.

Estimated values must therefore be used to completely define these automatically derived reaction networks. The estimations by analogy are approximations of unknown parameters by referencing known values from related molecules or groups. The other is using quantum calculations and applying statistical mechanics to the result to derive the estimates.

\paragraph{Thermodynamics}\mbox{}\\
Thermodynamic properties can be estimated by analogy using Benson group increment theory. The theory applies derived values for groups to the calculation of thermodynamic properties such as heat of formation, and entropy. A group contains an atom and its ligands. A species is fragmented into its groups, and a thermodynamic property is determined by the summation of the group contributions to that thermodynamic property. Group contributions have been determined for compounds containing a variety of atoms, and can also be applied to free radical intermediate species. This method depends on the availability of the group values, and there are continued efforts to determine new ones.

Calculation of thermodynamic properties via quantum chemistry increasingly becomes possible for automatic network generators due to the continuing advances in computing power. Instances where experimental data are unavailable and group estimations are not possible, these calculations can generate the required thermodynamic properties. A stable species has an energy minimum, and any perturbation in bond length or bond angle will increase the energy of the species. The geometry of the species must be optimized to this energy minimum, and the vibrational frequencies of this geometry calculated. These frequencies are used to generate the total partition function of the molecule, from which thermodynamic properties are derived (See section on Statistical Mechanics). This methodology has a higher computational cost than group additivity due to the complexity of the quantum chemical calculations.

\paragraph{Kinetics}\mbox{}\\
Kinetic parameters can be estimated on-the-fly by analogy, using the know kinetic parameters for a reaction and applying them to a reaction with undefined rate parameters. This is possible as the potential energy surfaces are similar for reactions of a like nature. Corrections can be applied to increase the accuracy of these estimations, and these corrections improve as more data is collected. This means that the kinetics can be quickly estimated for reaction families with abundant data with insignificant error. For example, hydrogen abstraction reactions have been extensively studied due to their prominence in combustion mechanisms, therefore these on-the-fly estimations work well for this family of reactions.

Estimations can be made for all reactions once the necessary parameters can be obtained from a related reaction, but the accuracy of these estimations depends on how similar the related reaction is, and also how many related reactions exist. Kinetics for reactions with less correlation data can be derived via transition state theory. This requires the definition of the transition state, which defines the minimum energy pathway for the reaction. There is a high computational cost associated with this method, which increases with the desired accuracy due to the level of theory used.

\subsubsection{Reaction Classes}
Reaction mechanisms are often explained by drawing arrows to signify the movement of electrons, but this cannot be used computationally to define a reaction pathway. Early attempts to create these classes involved matrix combinations and 

\subsubsection{EXGAS}
\subsubsection{MAMOX}
\subsubsection{REACTION}
\subsubsection{Rule Input Network Generator (RING)}
\subsubsection{Reaction Mechanism Generator (RMG)}

The constant rise in carbon dioxide levels has been attributed to the industrial advances. These advances have largely been driven by the combustion of fossil fuels, that produces carbon dioxide as a major waste product. The 

\subsection{Molecular Geometry}

Molecules need to be represented in three dimensional space before electronic structure optimization can take place. There are different methods to represent the coordinates of the atoms, as well as methods to convert two dimensional molecular graph representations to three dimensional structures.

\subsubsection{Geometry Representation}

These methods describe molecules in three dimensional space. From these representations, the molecules can be pictorially represented, and altered to optimize the molecular energies.

\paragraph{Cartesian Coordinates}\mbox{}\\

The atoms of a molecule are assigned coordinates in three dimensional space based on a cartesian axis. Each atom requires their full three point coordinate in order to be positioned. The atoms are, therefore, completely independent of each other.

\paragraph{Internal Coordinates}\mbox{}\\

The atoms of a molecule are represented relative to each other, based on distance, angle, and dihedral angle. The first atom is the reference atom. The second atom must be assigned a distance from the first to be positioned. The third atom requires its distance from the second atom, as well as the angle between the lines connecting itself and the second atom, and the second and first atoms. Except for these three atoms, all other atoms require their distances relative to an atom, an angle to another atom, as well as a dihedral angle relative to a third.

\subsubsection{Three Dimensional Generation}

Methods to represent the molecule correctly in three dimensional space once provided the atoms and bonds have been developed to aid the geometry generation. 

\paragraph{Distance Geometry}\mbox{}\\

Chemical rules can be applied to determine the atom distances. Based on orbital hybridizations, bonded atom maximum and minimum distances can be set. Also, atoms not bonded to each other but reside within the same molecule have a minimum distance associated with the van der Waals values, and maximum distance based on their connectivity through the molecule. Atoms on separate molecules have minimum separations based on the van der Waals distances, and infinitely large maximum distances. These rules can help generate possible three dimensional structures.

\paragraph{Rule-based Geometry}\mbox{}\\

Empirical rules have been developed and can be applied to molecules to determine their geometries. These rules have estimated bond distance and angles for various atom to atom bonds and hybridizations. For example, naphthalene can be derived by the rules of benzene as it is essentially two fused benzene rings. However, It would not be prudent to apply the rules of carbon-carbon single and double bonds to naphthalene, as this would neglect the effects of the aromatic pi bonds.

\subsection{Computational Tools}
\subsubsection{Distance Geometry}
\paragraph{RDKit}
\subsubsection{Quantum Chemistry}
\paragraph{Gaussian}
\paragraph{MOPAC}
\subsection{Transition State Searches}
\subsubsection{Surface Walking}
\subsubsection{Double-Ended Searches}
\section{Preliminary Results}
talk about H abstraction

\section{Experimental Design and Methods}

Below are the aims for my thesis.

\subsection{Aim 1: Automatic transition state searches from the reactant and product geometries.}
\subsubsection{Rationale}

Kinetic parameters for mechanism generation are ideally obtained from experimental data. Due to the difficulties in measuring intermediate reaction rates, experimental data are often sparse. Group additivity methods can be applied to estimate unknown rates, but fail when data are unavailable for analogous reactions. Statistical mechanics can be applied via transition state theory to calculate rate parameters, but require transitions state geometries. These geometries are obtained by deriving an estimate and optimizing it to a first order saddle point along the potential energy surface. These need to be generated automatically in order to be coupled with a reaction mechanism generator.

\subsubsection{Experimental Plan}

Distance geometry 

\subsubsection{Interpretations}
\subsubsection{Pitfalls}
\subsubsection{Alternative}

\subsection{Aim 2: Calculate kinetic parameters from the transition state geometries via application of transition state theory.}
\subsubsection{Rationale}

Generation of transition state geometries is of chemical interest as it provides mechanistic insight to a reaction. From the perspective of mechanism generation, it is only the first step in determining the desired rate parameters. The optimized geometries of the transition state and the reactants can be used to derive the necessary properties to calculate the rate parameters via transition state theory. These required the derivation of the total partition functions as well as information about their positions on the potential energy surface.

\subsubsection{Experimental Plan}
\subsubsection{Interpretations}
\subsubsection{Pitfalls}
\subsubsection{Alternative}

\subsection{Aim 3: Build the detailed combustion model for butyl benzene.}



\subsubsection{Rationale}
\subsubsection{Experimental Plan}
\subsubsection{Interpretations}
\subsubsection{Pitfalls}
\subsubsection{Alternative}

%\subsection{Aim 4: }
%\subsubsection{Rationale}
%\subsubsection{Experimental Plan}
%\subsubsection{Interpretations}
%\subsubsection{Pitfalls}
%\subsubsection{Alternative}

\section{Conclusions}

\end{document}

\bibliography{}
