\documentclass[12pt]{article}
\begin{document}

\title{}
\author{Pierre L. Bhoorasingh}
\date{\today}
\maketitle

\newpage
\tableofcontents

\newpage

\section{Summary}

Evans-Polanyi application and improvements
It has been applied to various reaction types \cite{Wijaya:2003di} \cite{Saeys:2003ku} \cite{Yaluris:1995hc}and even improved upon for specific applications \cite{Blowers:jx}.

\subsection{Aims}
Aim 1: Automate transition state searches for hydrogen abstraction reactions from the reactant and product geometries.
Aim 2: Automate kinetic parameter calculation from the transition state geometries via application of transition state theory.
\section{Introduction}

Combustion of liquid fuels is expected to remain the main source of energy, especially as transportation fuel, for the foreseeable future\cite{Ghoniem:2011iz}. Wedge theory postulates combining multiple solutions to eliminate the environmental impact of these fuels\cite{Pacala:2004hd}. Renewable biofuels offer an alternative liquid energy source to reduce our dependence of fossil fuels. Advanced engines, such as reactivity controlled compression ignition (RCCI) engines \cite{Kokjohn:dh} and homogeneous charge compression ignition (HCCI) engines \cite{Yao:2009ic}, offer higher efficiencies and lower emissions.

As each biofuel has unique combustion chemistry, and these advanced engines depend heavily on that chemistry, detailed kinetic models describing the elementary reactions are required to understand and make reliable predictions at conditions of interest which often lie outside safe experimental settings.

These models can contain upwards of 20,000 reactions, therefore automatic mechanism generators, such as Reaction Mechanism Generator (RMG), offers a practical method to build these networks. Reaction families dictate which bonds are formed, broken, or altered given specific types of atoms, bonds, and functional groups are present. Chemical graph theory \cite{Trinajstic:1991uo} is applied in the automatic generator to apply the rules of the reaction families to all species, generating a detailed and complex network of reactions.

Thermodynamic and kinetic properties are required to characterize the system of reactions, but thermodynamic data values are scarce, and even less kinetic parameters are known. Therefore, estimation methods are used to generate the required properties.

Benson group additivity \cite{Benson:1976wu} is used quickly calculate thermodynamic properties for species with unknown data. Molecules are decomposed into groups that have predetermined contributions to thermodynamic properties, but not all molecular groups have been characterized such as multi-ringed species. Quantum chemistry has been used to optimize the species geometry then evaluate the thermodynamic properties via statistical mechanics.

Rate parameters can be estimated by analogy to reactions with the same moiety \cite{Sumathi:2002gm}. These correlations, such as the Evans-Polanyi principle \cite{Evans:1938dl}, relate thermodynamic properties to the difference in activation energy of a reaction with known parameters to one that is unknown. Fundamental to this is the reaction coordinate for these reactions should behave similarly so that the changes in entropy are relatively similar or proportional and will be captured in the thermodynamic relationship. This is not always pertinent \cite{Hemelsoet:hk} and assumes data is available for analogous reactions, which is often not the case. As such, very approximate rate parameters are often determined using these estimation methods.

Alternatively, transition state theory (TST) is used to determine rate parameters by first finding the transition state (TS), the geometry located at the first order saddle point along the reaction minimum energy pathway. The theory suggests the TS is an activated complex that decomposes to the products while in equilibrium with the reactants. Correct description of the molecular properties of the reactants and TS enables calculation of the rate. Quantum chemistry programs optimize these geometries to provide this information, but require geometry estimates as inputs. Postulating the TS can be done manually or by using double-ended search algorithms which require reactant and product geometries positioned in a specific alignment. With reaction mechanisms containing a few thousand unique reactions with unknown rates, building the mechanism this way becomes tedious.

Formerly a barrier to applying TST is the computational cost of optimizing the geometries, but with the advances in high performance computing it is an optimal moment to explore the automation of transition state geometry generation. This would enable kinetic calculations for key reactions with unknown rates improving the accuracy of the detailed networks generated.



\section{Critical Literature Review}
\subsection{Automatic Mechanism Generation}



\subsubsection{Network Generation}
Automatic network generation systems involve methods that teach chemistry to a computer. These automatic systems all have common features. They must have a method to represent atoms, bonds, and other molecular properties. There must be transformation methods to represent reactions in order to convert reactants to products and methods to determine if a molecule generated is unique from all previously generated species. These will enable the generation of all the possible reactions.

\paragraph{Possibly talk about chemical graph theory, SMILES, InChI, etc.}

\subsubsection{Thermodynamic and Kinetic Parameters}
With the chemical reaction network defined, the thermodynamic and kinetic parameters are needed to fully describe the reaction system. Also, model reduction may also be desired due to the scale of these mechanisms, and kinetic parameters are required to conduct sensitivity analyses.

Describing thermodynamic and rate parameters by hand is taxing, since reaction mechanisms can have a very large number of species and reactions. Computational methods allow for the integration of the various methods to procure and specify these parameters for such large mechanisms. A ranking has to be implemented to ensure preferred methods are attempted before the less desirable, and this ranking is intended to maximize accuracy while minimizing computational cost. There are three general sources of data, both for kinetic and thermodynamic parameters: experimental values, estimation by analogy, and estimation via quantum mechanical calculations.

Well-founded experimental values in databases are the premier data source. Not only do these parameters contain little error, but the high efficiency in extracting a value from a database suits automated systems. However, these coveted values are scare due to the system restrictions and experimental difficulties. For kinetics, most of the species in a mechanism are intermediates, making their detection and measurement and arduous task. Added to the difficulty is the complexity of the reaction environment. These two points lead to the scarcity in kinetic parameters. More data is available for thermodynamic properties, but these data are restricted to the system studied. Therefore, values for stable molecules are accessible, but species such as radical have limited data.  As such, it is not possible to derive a detailed mechanism using experimentally derived values.

Estimated values must therefore be used to completely define these automatically derived reaction networks. The estimations by analogy are approximations of unknown parameters by referencing known values from related molecules or groups. The other is using quantum calculations and applying statistical mechanics to the result to derive the estimates.

\paragraph{Thermodynamics}\mbox{}\\
Thermodynamic properties can be estimated by analogy using Benson group increment theory. The theory applies derived values for groups to the calculation of thermodynamic properties such as heat of formation, and entropy. A group contains an atom and its ligands. A species is fragmented into its groups, and a thermodynamic property is determined by the summation of the group contributions to that thermodynamic property. Group contributions have been determined for compounds containing a variety of atoms, and can also be applied to free radical intermediate species. This method depends on the availability of the group values, and there are continued efforts to determine new ones.

Calculation of thermodynamic properties via quantum chemistry increasingly becomes possible for automatic network generators due to the continuing advances in computing power. Instances where experimental data are unavailable and group estimations are not possible, these calculations can generate the required thermodynamic properties. A stable species has an energy minimum, and any perturbation in bond length or bond angle will increase the energy of the species. The geometry of the species must be optimized to this energy minimum, and the vibrational frequencies of this geometry calculated. These frequencies are used to generate the total partition function of the molecule, from which thermodynamic properties are derived (See section on Statistical Mechanics). This methodology has a higher computational cost than group additivity due to the complexity of the quantum chemical calculations.

\paragraph{Kinetics}\mbox{}\\
Kinetic parameters can be estimated on-the-fly by analogy, using the know kinetic parameters for a reaction and applying them to a reaction with undefined rate parameters. This is possible as the potential energy surfaces are similar for reactions of a like nature. Corrections can be applied to increase the accuracy of these estimations, and these corrections improve as more data is collected. This means that the kinetics can be quickly estimated for reaction families with abundant data with insignificant error. For example, hydrogen abstraction reactions have been extensively studied due to their prominence in combustion mechanisms, therefore these on-the-fly estimations work well for this family of reactions.

Estimations can be made for all reactions once the necessary parameters can be obtained from a related reaction, but the accuracy of these estimations depends on how similar the related reaction is, and also how many related reactions exist. Kinetics for reactions with less correlation data can be derived via transition state theory. This requires the definition of the transition state, which defines the minimum energy pathway for the reaction. There is a high computational cost associated with this method, which increases with the desired accuracy due to the level of theory used.

\subsubsection{Reaction Classes}
Reaction mechanisms are often explained by drawing arrows to signify the movement of electrons, but this cannot be used computationally to define a reaction pathway. Early attempts to create these classes involved matrix combinations and 

\subsubsection{EXGAS}
\subsubsection{MAMOX}
\subsubsection{REACTION}
\subsubsection{Rule Input Network Generator (RING)}
\subsubsection{Reaction Mechanism Generator (RMG)}

The constant rise in carbon dioxide levels has been attributed to the industrial advances. These advances have largely been driven by the combustion of fossil fuels, that produces carbon dioxide as a major waste product. The 
\subsection{Molecular Geometry Optimization}

Molecules need to be represented in three dimensional space before electronic structure optimization can take place. There are different methods to represent the coordinates of the atoms, as well as methods to convert two dimensional molecular graph representations to three dimensional structures.

\subsubsection{Geometry Representation}

These methods describe molecules in three dimensional space. From these representations, the molecules can be pictorially represented, and altered to optimize the molecular energies.

\paragraph{Cartesian Coordinates}\mbox{}\\

The atoms of a molecule are assigned coordinates in three dimensional space based on a cartesian axis. Each atom requires their full three point coordinate in order to be positioned. The atoms are, therefore, completely independent of each other.

\paragraph{Internal Coordinates}\mbox{}\\

The atoms of a molecule are represented relative to each other, based on distance, angle, and dihedral angle. The first atom is the reference atom. The second atom must be assigned a distance from the first to be positioned. The third atom requires its distance from the second atom, as well as the angle between the lines connecting itself and the second atom, and the second and first atoms. Except for these three atoms, all other atoms require their distances relative to an atom, an angle to another atom, as well as a dihedral angle relative to a third.

\subsubsection{Three Dimensional Generation}

Methods to represent the molecule correctly in three dimensional space once provided the atoms and bonds have been developed to aid the geometry generation. 

\paragraph{Distance Geometry}\mbox{}\\

Chemical rules can be applied to determine the atom distances. Based on orbital hybridizations, bonded atom maximum and minimum distances can be set. Also, atoms not bonded to each other but reside within the same molecule have a minimum distance associated with the van der Waals values, and maximum distance based on their connectivity through the molecule. Atoms on separate molecules have minimum separations based on the van der Waals distances, and infinitely large maximum distances. These rules can help generate possible three dimensional structures.

\paragraph{Rule-based Geometry}\mbox{}\\

Empirical rules have been developed and can be applied to molecules to determine their geometries. These rules have estimated bond distance and angles for various atom to atom bonds and hybridizations. For example, naphthalene can be derived by the rules of benzene as it is essentially two fused benzene rings. However, It would not be prudent to apply the rules of carbon-carbon single and double bonds to naphthalene, as this would neglect the effects of the aromatic pi bonds.

\subsection{Transition State Theory}
\subsubsection{Classical TST, RC-TST, V-TST}

\subsection{Computational Tools}
\subsubsection{Distance Geometry}
\paragraph{RDKit}
\subsubsection{Quantum Chemistry}
\paragraph{Gaussian}
\paragraph{MOPAC}
\subsection{Transition State Searches}
\subsubsection{Surface Walking}
\subsubsection{Double-Ended Searches}
\section{Preliminary Results}
talk about H abstraction

\section{Experimental Design and Methods}

Below are the aims for my thesis.

\subsection{Aim 1: Automate transition state searches for hydrogen abstraction reactions from the reactant and product geometries.}
\subsubsection{Rationale}

Kinetic parameters for automatic mechanism generation are ideally obtained from experimental data, but are often sparse due to difficulties in measuring reaction intermediates. Group additivity methods can be applied to estimate unknown rates, but lose accuracy when data are unavailable for analogous reactions.

Transition state theory can be used to calculate rate parameters, but require the correct geometries of the transition state. For practical application in a automatic network generator, the transition state geometry of a reaction must be derived automatically. Without previous knowledge of the transition state, other methods must be used to derive the geometry. Double-ended methods have been developed to estimate the transition state geometry using the geometries of the reactants and products. Each method uses their own interpolation strategy with varying levels of success.

These double-ended methods require positioning of the reactive atoms in a state prior to the reaction. For hydrogen abstraction and disproportionation, the three reactive atoms are the abstracted hydrogen, the atom bonded to the hydrogen, and the atom abstracting the hydrogen. These need to be positioned with the abstracted hydrogen between the other two, and with consideration of the distances. The proposal is to use distance geometry to position the reacting species, thereby configuring the double-ended searches.

\subsubsection{Experimental Plan}

A distance geometry matrix can be generated in RDKit describing the molecular geometries of the reactants. As no bond connects the two reactants, the matrix will consist very large numbers for the upper bounds describing the distances between two atoms on the separate molecules. Embedding a molecules this way will position the atoms infinitely apart.

The distance between the abstracted hydrogen and the abstracting radical atom can be defined by editing the limits of the distance defined in the matrix. Manual testing has shown a range of 2.0 to 2.1 {\AA} as a good starting point for this distance. It has also shown that this is not enough to properly position the reacting species. The distance between the abstracting radical atom and the atom bonded to the abstracted hydrogen also needs to be defined at approximately 2.5 to 2.6 {\AA}.

Once the bounds matrix is edited, embedding and optimizing the geometry within the limits of the matrix generates a reactant geometry with the abstracting radical atom approximately 2.0 {\AA} from the abstracted hydrogen. The same can be done for the product geometry, except the abstracting radical atom is now the atom bonded to the abstracted hydrogen, and vice versa.

These geometries can be automatically printed and sent to a computational chemistry program to conduct a double-ended search. Each computational chemistry software has its own syntax for input, but a protocol to generate these input files for most software packages has been created in the open-source CCLib. This will return a transition state estimate that will be optimized to the transition state geometry.

\subsubsection{Interpretations}

The generated transition state geometry should have a single imaginary frequency representing the reaction path. The resulting saddle point geometry can be verified by exploring the path along this negative frequency in both directions from the transition state. Successful exploration of this path should find the reactant and product geometries, which can be compared to the original inputs thereby verifying the correct transition state has been found.

\subsubsection{Pitfalls}

The reactant and product molecular pairs are positioned within a small range of gaps using distance geometry. A set range will be used to setup the calculations for all possible hydrogen abstraction reactions. But interatomic distances are related to the interatomic forces, so it is possible that these distances will not work for all possible reacting pairs. For instance, a methyl radical does not need to be as close to the abstracted hydrogen as a hydrogen radical.

The double-ended methods were designed for reactions with complex geometries, therefore it is uncertain how these will work with simpler reactions (small molecule reactions) where the transition states are easily determined. Early indications show a large percentage of these reactions to be problematic, such as reactions involving a hydrogen radical as a reactant or product species. This may also be related to the standard distances used when editing the bounds matrix.

Verification of the resulting geometry uses the IRC calculations. This provides the two geometries on either side of the transition state without any bonding information. Bonds are first estimated based on distance, then bond orders are set based on valencies. If the path corresponding to the reaction path has not been completely explored, the bond estimation may be incorrect leading to cases where the right transition state is determined to have failed.

\subsubsection{Alternative}

Using standard distances for all reactions is not expected to work for all cases. Adjustments to these distances can be made depending on the atom types involved in the reaction as the interatomic forces will determine the geometry of the transition state.

The generated geometries will be placed in a database which can assist in future transition state searches. Over many years of chemistry research, reaction transition states have been determined manually and used to determine kinetic parameters. Often very little data about the geometries are included in the paper, but are often included in supplementary materials. These can supplement the geometries found via the automatic method.

\subsection{Aim 2: Automate kinetic parameter calculation from the transition state geometries via application of transition state theory.}
\subsubsection{Rationale}

From the perspective of mechanism generation, automatic transition state geometry generation is only the first step in determining the desired rate parameters. The optimized geometries of the transition state and the reactants can be used to derive the rate parameters via transition state theory. These required the derivation of the total partition functions as well as information about their positions on the potential energy surface. When the required geometries are optimized, the molecular energy modes are also calculated and can be used to calculate the total partition function.

\subsubsection{Experimental Plan}

Classical transition state theory uses the reactant and transition state partition functions to generate rate parameters. The rate parameters are found in Arrhenius form, which consists of an 'A' value, and an exponential term with a larger temperature dependence. 
\subsubsection{Interpretations}
\subsubsection{Pitfalls}
\subsubsection{Alternative}

\subsection{Aim 3: Implement direct transition state estimation from generated database.}
\subsubsection{Rationale}



\subsubsection{Experimental Plan}
\subsubsection{Interpretations}
\subsubsection{Pitfalls}
\subsubsection{Alternative}

%\subsection{Aim 4: }
%\subsubsection{Rationale}
%\subsubsection{Experimental Plan}
%\subsubsection{Interpretations}
%\subsubsection{Pitfalls}
%\subsubsection{Alternative}

\section{Conclusions}

\end{document}

\bibliography{}

